\begin{abstract}
    Axelar network is a decentralized interoperability layer that connects a
    large set of blockchains. 
    %
    Axelar is based on the Cosmos SDK and defines a set of Tendermint
    proof-of-stake validators, who can relay events from supported chains to
    the Axelar Cosmos side.
    In particular, Axelar
    %
    leverages multiple validation models for events happening on the supported
    EVM chains. Such events can be reported via IBC light clients, when
    possible, or attestations from Axelar's proof-of-stake validators.
    %
    Axelar validators are expected to run full nodes of the EVM chains to
    gather event information. However, this operation is typically expensive, 
    such that the entry and maintenance costs for validators increase, along with
    possibly leading some validators to use centralized RPC providers.
    %
    This proposal aims to reduce operational costs and increase Axelar's 
    level of decentralization. Specifically, we advocate for an on-chain light client to run
    as an Axelar module that trustlessly consumes source data.
    Such a light client checks the veracity of cross-chain claims to ensure the
    relevant events have been included in the canonical source chain,
    safeguarding against untrustworthy EVM RPC providers.
    %
    Focusing on Ethereum, we discuss and contrast several possible flavors of
    light or superlight clients based on \emph{Simple Payment Verification}
    (SPV), \emph{zero knowledge}, or \emph{refereed games}. Tailoring the
    solution to the particularities of Axelar, we recommend SPV as the most
    suitable choice. 
    %
    We present an architecture blueprint and estimate the workforce allocation
    as well as the cost needed for the effort.
    % 
    Lastly, we put forth a theoretical model to support the need for such
    design changes. Theoretically, our proposed changes enable the system to
    endure temporary dishonest advantages among Axelar validators and restore
    safety following the re-establishment of an honest supermajority.
\end{abstract}
