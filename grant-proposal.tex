\section{Grant Proposal}

We plan to implement this project in the course of $4-6$ months. Anyone from the Common Prefix group
can be involved on an as-needed basis depending on the expertise required. For example, the following
people will likely be relevant to the implementation of this project:

\begin{itemize}
    \item \textbf{Dionysis Zindros} is a post-doctoral researcher at Stanford University. His research is focused on light and superlight clients, on which he did his Ph.D. dissertation, “Decentralized Blockchain Interoperability.” Some of his relevant published works include \href{https://eprint.iacr.org/2017/963.pdf}{Non-Interactive Proofs of Proof-of-Work}, \href{https://eprint.iacr.org/2018/1048.pdf}{Proof-of-Work Sidechains}, \href{https://eprint.iacr.org/2018/1048.pdf}{Proof-of-Stake
      Sidechains}, \href{https://eprint.iacr.org/2021/623.pdf}{Mining in Logarithmic Space}, \href{https://arxiv.org/abs/2203.15968}{Light Clients for Lazy Blockchains}, \href{https://eprint.iacr.org/2019/1096.pdf}{Proof of Burn}, \href{https://eprint.iacr.org/2020/1122.pdf}{The Velvet Path to Superlight Blockchain Clients}, \href{https://eprint.iacr.org/2019/1444.pdf}{Compact Storage of Superblocks for NIPoPoW Applications}, \href{https://eprint.iacr.org/2020/138.pdf}{Smart Contract Derivatives}, and
      \href{https://eprint.iacr.org/2020/927}{A Gas-Efficient Superlight Bitcoin Client in Solidity}. He co-authored the “Proofs of Proof of Proof of Stake in Sublinear Complexity” paper. He has published in top peer-reviewed conferences, such as IEEE Security \& Privacy, ESORICS, Financial Cryptography (and the Workshop on Trusted Smart Contracts), and Advances in Financial Technologies. Regarding practical software engineering experience, Dionysis has worked on the Product Security team at Twitter and the Incident Response Development team at Google. He has also presented at practical security conferences like Black Hat Europe and Black Hat Asia. 
  \item \textbf{Shresth Agrawal} is a developer, researcher, and entrepreneur. He is one of the authors of the “Proofs of Proof of Stake in Sublinear Complexity” paper and was responsible for building the first ever light client for Ethereum, Kevlar. He has experience building efficient and secure algorithms, protocols, and smart contracts for several DeFi protocols. Previously, he worked at ParaSwap, where he was responsible for architecting and developing a large portion of the core aggregation algorithm. He was awarded by the President of India for his research on contagious diseases. Shresth completed his Bachelor’s degree from Jacobs University Bremen and is currently pursuing his Master’s degree at Technical University Munich. He is interested in Cryptography, Security, Consensus Protocols, Decentralized Finance, and Ethereum.
  \item \textbf{Apostolos Tzinas} is a smart-contracts software engineer. He is pursuing a joint Bachelor’s and Master’s in Electrical and Computer Engineering at the National Technical University of Athens. Apostolos has extensive front-end software engineering experience working at Maya Insights and NutriDice, where he took on a full-stack software engineering role. He has worked with numerous programming languages and technical stacks. He loves algorithms, and as a high school student, he participated in Informatics Olympiads, such as the Junior Balkan Olympiad in Informatics. 
  \item \textbf{Dimitris Lamprinos} is a software engineer based in Thessaloniki. He holds a Bachelor’s degree in Computer Science from the Aristotle University of Thessaloniki. Dimitris works on smart contract development and basic consensus development. He has significant experience in developing and scaling web2 applications in Amondo, Geekbot, and Vidpulse. He also has algorithmic cryptocurrency trading experience and has worked on developing and deploying smart contracts to the Ethereum
    blockchain since the early days of ICO boom.
\end{itemize}

This project will be executed in multiple phases which are described in detail below.

\subsection{Phase 1: Architectural Design}
During this phase, we will study the Axelar codebase in depth and propose a suitable architecture consisting of a Go Cosmos module that will implement the on-chain Ethereum light client.
We will explore avenues keeping the following principles in mind: 
\begin{itemize}
  \item \textbf{Respectful of previous work.} We will propose an architecture that requires minimal changes to the codebase while maintaining the existing design choices of the codebase.
  \item \textbf{Extensibility.} The architecture will be designed in a fashion that will allow extending the light client to different source chains beyond Ethereum in the future  
  \item \textbf{Modularitity.} Additionally, the architecture will be modular in a way that allows for upgrading to a different proof mechanism (e.g., SPV or ZK) so that the protocol is future-proof.
\end{itemize}
\noindent
\textbf{Deliverables.} At the conclusion of this phase, we will deliver the following:

\begin{itemize}
  \item A detailed design document that describes the architecture of the light client to be implemented on Axelar. The design document contains details about all the different components that we will develop, including the Go module which will run within the Axelar codebase (taking the role of the verifier), as well as the modifications needed on the attestor codebase (also part of the full node) so that attestations can be reported to the on-chain verifier.
  \item The above report will be delivered in LaTeX PDF format and include pseudocode for all the relevant components so that their role is also clear from a theoretical point of view.
\end{itemize}

\subsection{Phase 2: Consensus Validation}

During this phase, we will implement the consensus components of the on-chain light client. Depending on the details discovered during the Architectural Phase, the implementation phase may be altered accordingly. Based on our current understanding of the Axelar architecture, during the consensus implementation phase we will need to implement the following:

\begin{itemize}
  \item Verification of the Ethereum sync committee \emph{handover signatures}, including aggregate signature verification and quorum logic.
  \item Verification of latest finalized block using sync committee signatures and Ethereum finalization logic.
\end{itemize}

\noindent
\textbf{Deliverables.} At the conclusion of this phase, we will deliver the following:

\begin{itemize}
  \item A Cosmos Go module that implements the consensus components of the on-chain Ethereum light client.
  \item A test suite that verifies the correctness of the consensus components.
  \item A component that allows any friendly-but-untrusted party to submit the consensus data of attestations (sync committee signatures and block headers) to the on-chain verifier. These will be produced using the Ethereum Beacon chain API.
\end{itemize}

\subsection{Phase 3: Execution Validation}
% TODO(shresth): High level description of this phase

During this phase, we will implement the execution logic of the on-chain Ethereum light client. This portion is responsible for checking EVM-related data such as Ethereum transactions and events.

\begin{itemize}
  \item Verification of Ethereum execution data, given a validated finalized Ethereum block. This includes the verification of transactions and events.
  \item This verification includes checking the validity of the Merkle proofs for transactions and/or events as needed.
\end{itemize}

\noindent
\textbf{Deliverables.} At the conclusion of this phase, we will deliver the following:

\begin{itemize}
  \item An extension of the previous Go Cosmos module that implements the execution components of the on-chain Ethereum light client.
  \item A test suite that verifies the correctness of the execution components of the Go module.
\end{itemize}

\subsection{Phase 5: Theory, Deployment \& Audit}
During this phase we will formalize the lightclient construction and prove its security guarantees. We will also deploy the light client to the Axelar's testnet and handover the codebase with documentation to the Axelar team. We also expect that the Axelar team will conduct an audit of the codebase before deploying it to the mainnet. During this phase we help the Axelar team with the audit process and fix any issues that may arise.

\noindent
\textbf{Deliverables.} At the conclusion of this phase, we will deliver the following:
\begin{itemize}
  \item Testnet deployment of the Ethereum light client to the Axelar network (Lisbon).
  \item Handover of the codebase to the Axelar team with proper documentation.
  \item A LaTeX document that formalizes the light client construction and proves its security guarantees.
  \item Any fixes to the codebase that may be required as a result of the audit.
\end{itemize}
