\section{Grant Proposal}

We plan to implement this project in the course of $4-6$ months. Anyone from the Common Prefix group
can be involved on an as-needed basis depending on the expertise required. For example, the following
people will likely be relevant to the implementation of this project:

% TODO(shresth): add a description of each person's expertise and their role on this project
\begin{itemize}
  \item \textbf{Shresth Agrawal}
  \item \textbf{Dionysis Zindros}
  \item \textbf{Apostolos Tzinas}
  \item \textbf{Dimitris Lamprinos}
\end{itemize}

This project will be executed in $3$ phases which are described in detail below.

\subsection{Phase 1: Architectural Design}
During this phase, we will study the Axelar codebase in depth and design a suitable architecture in the form of a Go Cosmos module that will implement the Ethereum light client functionality on-chain.
We will explore designs that minimally alter the existing Axelar codebase and respect the design principles of the ecosystem.
The architecture will be designed in a fashion that will allow extending the light client to different source chains beyond Ethereum. Additionally, the architecture will be modular in a way that allows swapping the proof mechanism (e.g., SPV or ZK) so that it is possible to experiment with different proof mechanisms in the future.

\noindent
\textbf{Deliverables.} At the conclusion of this phase, we will deliver the following:

\begin{itemize}
  \item A detailed design document that describes the architecture of the light client to be implemented on Axelar. The design document contains details about all the different components that we will develop, including the Go module which will run within the Axelar codebase (taking the role of the verifier), as well as the modifications needed on the attestor codebase (also part of the full node) so that attestations can be reported to the on-chain verifier.
  \item The above report will be delivered in LaTeX PDF format and include pseudocode for all the relevant components so that their role is also clear from a theoretical point of view.
\end{itemize}

\subsection{Phase 2: Consensus Validation}

During this phase, we will implement the consensus components of the on-chain light client. Depending on the details discovered during the Architectural Phase, the implementation phase may be altered accordingly. Based on our current understanding of the Axelar architecture, during the consensus implementation phase we will need to implement the following:

\begin{itemize}
  \item Verification of the Ethereum sync committee \emph{handover signatures}, including aggregate signature verification and quorum logic.
  \item Verification of latest finalized block using sync committee signatures and Ethereum finalization logic.
\end{itemize}

\noindent
\textbf{Deliverables.} At the conclusion of this phase, we will deliver the following:

\begin{itemize}
  \item A Cosmos Go module that implements the consensus components of the on-chain Ethereum light client.
  \item A test suite that verifies the correctness of the consensus components.
  \item A component that allows any friendly-but-untrusted party to submit the consensus data of attestations (sync committee signatures and block headers) to the on-chain verifier. These will be produced using the Ethereum Beacon chain API.
\end{itemize}

\subsection{Phase 3: Execution Validation}
% TODO(shresth): High level description of this phase

During this phase, we will implement the execution logic of the on-chain Ethereum light client. This portion is responsible for checking EVM-related data such as Ethereum transactions and events.

\begin{itemize}
  \item Verification of Ethereum execution data, given a validated finalized Ethereum block. This includes the verification of transactions and events.
  \item This verification includes checking the validity of the Merkle proofs for transactions and/or events as needed.
\end{itemize}

\noindent
\textbf{Deliverables.} At the conclusion of this phase, we will deliver the following:

\begin{itemize}
  \item An extension of the previous Go Cosmos module that implements the execution components of the on-chain Ethereum light client.
  \item A test suite that verifies the correctness of the execution components of the Go module.
\end{itemize}

\subsection{Phase 4: Theory}
% TODO(shresth): High level description of this phase

\subsection{Phase 5: Deployment}
% TODO(shresth): High level description of this phase

% TODO(shresth): Talk about this with Sergei and consider whether this is part of the current grant or a future grant
\begin{itemize}
  \item Testnet deployment of the Ethereum light client to the Axelar network (Lisbon). This deployment can be scoped in a future grant proposal.
  \item Handover of the codebase to the Axelar team.
\end{itemize}