\section{On-Chain Light Client}

In this section, we explore on-chain light clients as a solution for
implementing a bridge between Axelar and Ethereum. First, we discuss the
high-level benefits of such construction. Following, we review multiple
candidate solutions for implementing on-chain light clients, before finally
offering our recommendation for this particular instance.

\subsection{Merits of an On-Chain Light Client}

There are various advantages in using an on-chain light client, compared to
relying on the committee of attestors.

\noindent\textbf{Safety.}
One of major benefit is the preservation of safety at all times, even if there
exists a dishonest majority in the Axelar chain. In particular, a light client
will not accept an incorrect statement about an Ethereum transaction,
regardless of the security state of Axelar's chain.

This is in contrast with the existing design. Currently, the client accepts a
statement as long as it is validated by the Axelar attestors. These attestors
are chosen based on their stake on Axelar's chain. However, if the majority of
Axelar's stake becomes dishonest temporarily, the set of attestors would also
become adversarially-controlled. Consequently, the adversary could sign an
incorrect statement about Ethereum's state, which the clients would accept.

Using an on-chain light client avoids this problematic scenario, since the
client accepts a statement about Ethereum's state only as long as it is
accompanied by a valid proof. This proof is generated either by Ethereum's
validators or in a cryptographically unforgeable manner (e.g., using
zero-knowledge), therefore it does not rely on Axelar's security.

\noindent\textbf{Healing.}
The preservation of safety opens the possibility of using a healing mechanism,
in order to recover from a temporary adversarial majority in Axelar. In
particular, in such case, Axelar participants could employ a social consensus
mechanism, in order to remove the offending parties and restore honest
majority. Since safety is guaranteed at all times, it would be straightforward
to continue after the healing process ends from the last block for which safety
holds.

- incentives \textbf{--- DK: not sure what this is about, so leaving it here for now ---}

\subsection{On-Chain Light Client Alternatives}\label{subsec:alternatives}

\subsubsection{SPV Light Client (sampling validators)}
\subsubsection{SPV Light Client (sync committee)} 
        - 24kb 
        - a sampling of validators vs sync committee
\subsubsection{Optimistic Light Client}
        - doesn't make sense due to high bridge traffic
\subsubsection{Bisection Light Client}
        - doesn't make sense due to high bridge traffic
\subsubsection{SNARK Light Client}
        - recursion is not helpful
        - groth16: 2 group elements
        - plonk
        - holo2
        -
\subsubsection{Nice table with a comparison}
         - accountability
         - interactivity
         - proof size 
         - safety assumptions
             - eth safety/liveness
             - axelar validators 
             - axelar attesters 
         - liveness assumptions
         - historical validity
         - implementation complexity
\subsubsection{Our recommendation}
